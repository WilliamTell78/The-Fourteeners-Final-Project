\documentclass[12pt]{article}
\usepackage{tocloft}
\usepackage{natbib}
\usepackage{url}
\usepackage[utf8x]{inputenc}
\usepackage{amsmath}
\usepackage{graphicx}
\usepackage{verbatim}
\graphicspath{{images/}}
\usepackage{parskip}
\usepackage{fancyhdr}
\usepackage{vmargin}
\setmarginsrb{3 cm}{2.5 cm}{3 cm}{2.5 cm}{1 cm}{1.5 cm}{1 cm}{1.5 cm}
\usepackage{appendix}
\usepackage{listings} % For code importing
\usepackage{xcolor} % for setting colors
\input{arduinoLanguage.tex}  


\begin{document}
\title{Project Report}
%%%%%%%%%%%%%%%%%%%%%%%%%%%%%%%%%%%%%%%%%%%%%%%%%%%%%%%%%%%%%%%%%%%%%%%%%%%%%%%%%%%%%%%%%

\begin{titlepage}
\centering
    \vspace*{0.5 cm}
    \includegraphics[scale = 0.11]{isu_seal.png}\\[1.0 cm] % University Logo
    \textsc{\LARGE IOWA STATE UNIVERSITY}\\[2.0 cm]
    \textsc{\large AEROSPACE ENGINEERING DEPARTMENT}\\[0.2 cm]
    \textsc{\large Computational Techniques for Aerospace Design}\\[0.2 cm]
\textsc{\Large AERE 361}\\[0.5 cm] % Course Code
\textsc{\Large Spring 2021}\\[0.5 cm] % Course Code
\textsc{\Large Final Project Report}\\[0.2 cm]
\textsc{\Large The Fourteeners}\\[0.2 cm]
\rule{\linewidth}{0.2 mm} \\[0.4 cm]
%{ \huge \bfseries \thetitle}\\


\begin{minipage}{0.8\textwidth}

\begin{flushleft}
\emph{Team Member Names :} \\
Cashen, William\linebreak
Hood, Delane\linebreak
Winter, Alex\linebreak
Kapoor, Khushi\linebreak
\end{flushleft}
\end{minipage}\\[2 cm]

\vfill

\end{titlepage}

%%%%%%%%%%%%%%%%%%%%%%%%%%%%%%%%%%%%%%%%%%%%%%%%%%%%%%%%%%%%%%%%%%%%%%%%%%%%%%%%%%%%%%%%%
%\maketitle
\tableofcontents
\pagebreak
%%%%%%%%%%%%%%%%%%%%%%%%%%%%%%%%%%%%%%%%%%%%%%%%%%%%%%%%%%%%%%%%%%%%%%%%%%%%%%%%%%%%%%%%%

\section{ABSTRACT}
Will
Will-Also put and format images into project.
This is your abstract.  It is a short summary of what your report will cover.  You should keep your abstract to 250 words or less.  Use this to ``hook in'' your reader.

\section{INTRODUCTION}
Khushi
The introduction is where you will introduce your group and your project. List out the team members (optionally include a picture) and what their role is. Briefly introduce your project, what it does and why. A minimum of 2 paragraphs. Introduction is worth 5 points.

\section{FEATURES}
There were 5 main features that were used for our project. This list of features changed drastically throughout the semester. The few other components we were originally hoping to have in the project were a 7-segment display, possibly using LEDs, and even the possibility of using another servo to control lateral movement of the kicker. 

The first step in our project is using a joystick to control Messi's kick. Instead of using it is a joystick however, we used the push sensor capability to automate the kick. The code for this was quite simple and was connected to the movement of the servo. 

Next, the servo was also a main feature of our project. As mentioned above, it controlled the kick by the push on the joystick. We originally did some research into which type of servo we would be needing to use. Even after our research, and trying to see what different setting in our code we could use, we were still running into some issues with the power of the kick so we ended up using some fishing wire to create some torque. 

The third component was the ultrasonic sensor. This sensor was placed at the goal line to detect whether or not the ball went inside the goal. This was a part of our original plan and we are glad it ended up working. We originally had to do some adjustments to make the sensor sensitive enough to pickup the ball moving in front of it. 

The pressure sensors were also an original part of our idea. We placed a pressure sensor in each corner of the goal. This is where a soccer player is most likely to score a goal, according to our research. We did a lot of research and adjustments to try and make these two sensors more sensitive as well, and found that changing the resistance really worked. 

Lastly, the neopixels on the CPX board served as our LEDs/7 segment display idea. Since we were not able to incorporate them due to a shortage of pin connections, we resorted to using the neopixels. As described in the solution, we used different use cases to light up the board with different colors. 


\section{PROBLEM STATEMENT}
Will
Your problem statement is stating what the problem(s) that you are attempting to solve. Again, this should not be a copy and paste from your proposal. State the problem and why you are solving it. This should be backed up with some light research. You may use the same references from your proposal, but if you done some more research since then, include additional citations as well. This section is worth 10 points.

\section{PROBLEM SOLUTION}
Delane
Here, you go into detail what your solution to the problem is.  I expect that this will have several subsections and you should breakout each area.  You should include any graphics and pictures as relevant as well and reference them like Figure \ref{fig:cpx}.
\begin{figure}[!t]
\centering
\includegraphics[width=4.5in]{cpx01.jpg}
\caption{This is the circuit playground express}
\label{fig:cpx}
\end{figure}

Again, cite any sources that you have.  If you took snippets of code or found a paper that discusses on how to do something, then you need to cite it. The same if you got inspiration for code from a source, cite that as well. For this final project report, I am expecting at least 3 sources cited.  One will probably be what you had in your problem statement from your proposal.  

Your problem solution is one of the largest things we look at. I am looking for the following items:

\begin{itemize}
    \item How did you come up with your solution
    \item How did you test or verify your solution
    \item Do you think this was a good solution?
    \item Show as much as you can of the solution in action (pictures and/or data)
\end{itemize}

For this reason, this section has the most points at 25 points

\section{STATUS}
Alex
Here, you need to honestly assess what the status of the project is.  If successful, state that it was successful and all the goals that it achieved (your goals are from your project proposal).  If not successful, state what was completed, what was not completed and state what happened. This part is worth 5 points

\subsection{Lessons Learned}
Khushi
Here, put any lessons learned from this project.  This may also relate to some of the items that you did not accomplish with this project. If you did not accomplish something, why? What might you do differently? I am also looking for what the group learned through this process. The obvious answer is ``programming'', but I am looking beyond that. Tell me what other skills you think that you learned or that you improved up working on the project. These can include ``soft'' skills like teamwork, communication, leadership, etc. This section is worth 10 points

\section{RESULTS}
 Alex
Put all results here.  If you collected data, explain and show at least some analysis on the data you collected.  If no data is collected, you should have collected reactions from others using your device and put that feedback here.  Any graphs you generated should be here as well.

You must include a copy of your source code in the appendix.  There is an example of this below.  Also, include the link to your GitHub repository.  You can use the \verb=\url=  command like this \url{https://github.com/AerE-361-FinalProject/Project-Report-AerE-361}. Make sure you reference where the code is located as well as any other data. This section is worth 10 points.


\section{FUTURE WORK}
As a team, we briefly discussed possible future work with Professor Nelson. As we described in our presentation, the 3 main things we would hope to work on are the mechanical structure of the soccer field, the servo motor used to kick, and the pressure sensors. For the structure of the field, we found that at times it was hard to recreate each kick due to the cardboard or maybe an imbalance in the field. This could be mitigated if we used a more sturdy field which was built from stronger materials and also made it a little bigger. Secondly, for the servo, we discussed possibly using a motor instead of a servo since that servo isn't really meant for a repetitive rotational motion and would have been better for maybe a translation motion. Lastly, we would work to see if we can do a calibration reset on the pressure sensors that averages the values it gets before every test so that it is calibrated and runs properly. 

\section{CONCLUSION}
Alex
Finally, wrap up your report. Although there is no points here, it is expected.

\newpage

\bibliographystyle{plain}
\bibliography{ref}
\newpage
% you need to have at least your code in your appendix
\appendix

\section{SOURCE CODE}
Alex-Put code in
Source Code
\lstinputlisting[language=Arduino]{src/demo.ino}
\end{document}
